\documentclass[12pt,a4paper,fleqn]{article}
\usepackage[left=1cm,right=1cm,top=2cm,bottom=2cm]{geometry}

\usepackage[T2A]{fontenc}
\usepackage[utf8]{inputenc}
\usepackage[russian,english]{babel}
\usepackage{amsmath}
\usepackage{amsfonts}
\usepackage{mathtools}
\usepackage{amssymb}
\usepackage{tikz}

\renewcommand{\labelenumii}{\arabic{enumi}.\arabic{enumii}.}
\everymath{\displaystyle}
\setlength\parindent{0pt}
\renewcommand\div{\ \vdots\ }
\newcommand\ndiv{\not\vdots\ }

\title{Задача 4. Длинное число} \author{Чубий Савва} \date{}

\begin{document}
\maketitle
\sloppy \hbadness 10000000

\begin{enumerate}
    \item 
        \begin{gather*}
            \begin{rcases}
                2^n \text{ оканчивается на } n\\
                2^n \div 2^9
            \end{rcases}
            \Rightarrow n \div 2^9
        \end{gather*}
        Т.е. убираем все $n \ndiv 2^9$.

    \item Осталось довольно мало чисел. Их можно просто перебрать.
    \item Для перебора будем использорвать алгоритм бинарного возведения в
        степень (по модулю $10^9$).
    \item После нахождения подходящего числа можно завершить программу, а можно
        продолжить её выполнение и выяснить, что ответ единственный.
\end{enumerate}
\textbf{Ответ.} $432948736$
\end{document}
